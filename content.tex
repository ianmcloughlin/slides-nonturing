\begin{frame}{Polynomial time}
  \begin{definition}
    An algorithm is said to be solvable in \emph{polynomial time} if the number of steps required to complete the algorithm for a given input is $O(n^k)$ for some nonnegative integer $k$, where $n$ is the complexity of the input.
  \end{definition}
  \vspace{4mm}
  \begin{alertblock}{Informally: P complexity class}
    The P complexity class is the set of problems for which there exists, for each such problem, at least one algorithm to solve that problem in polynomial time.
  \end{alertblock}
\end{frame}


\begin{frame}{Polynomial time on a Turing machine}
  \begin{description}
    \item[Sorting] algorithms are usually compared in terms of comparisons.
    \item[Other algorithms] might be compared in terms of something else, like iterations.
    \item[With Turing machines] we can use the number of times we look up the state table.
    \item[The size] of the input can be the length of the input on the tape initially.
  \end{description}
  \vspace{4mm}
  \begin{alertblock}{P complexity class}
    The P complexity class is the set of languages for which there exists some Turing machine that decides the language in polynomial time.
  \end{alertblock}
\end{frame}





\begin{frame}{Non-deterministic Turing machine}
  \begin{description}
    \item[The usual] Turing machines are often called deterministic Turing machines.
    \item[Deterministic] Turing machines have exactly one row in their state table for every combination of (non-terminal) state and tape symbol.
    \item[This means] there is only one path to follow at a given point in time.
    \item[Nondeterministic] Turing machines can have any number of rows for each state/symbol (including none).
    \item[Essentially] they allow for parallel computation -- they can branch into two or more paths at the same time.
  \end{description}
\end{frame}


\begin{frame}{Non-deterministic Turing machine and languages}
  \begin{description}
    \item[Languages] are accepted by non-deterministic Turing machines, where an input string is accepted if any branch ends in the accept state.
    \item[Deciders] -- if a non-deterministic Turing machine always halts on all branches of computation, no matter what the input, then we say it decides the language it accepts.
    \item[Any] language that is accepted (or decided) by a non-deterministic Turing machine has some deterministic Turing machine that accepts (or decides) it. So non-deterministic Turing machines don't really have any extra abilities over deterministic ones.
  \end{description}
\end{frame}


\begin{frame}{Non-deterministic polynomial time}
  \begin{definition}
    A decision problem is in the NP complexity class if it is decidable by a non-deterministic Turing Machine in polynomial time.
  \end{definition}
  
  \begin{alertblock}{P is a subset of NP}
    Note that every determinisitic Turing machine is also a non-deterministic one, by our definitions.
    The P complexity class is a subset of NP because of this.
  \end{alertblock}

  \begin{alertblock}{Equivalent definition}
    An equivalent definition of NP that you may come across is that NP is the set of languages $A$ that can be verified in polynomial time.
    By verified we mean that a deterministic Turing machine can accept a language $\{ wc \}$ where $w$ is in $A$ and $c$ is some string, called the certificate for $w$.
  \end{alertblock}

\end{frame}


\begin{frame}{NP-complete problems}
  \begin{definition}
    A problem is NP-hard if each problem in NP can be reduced to it in polynomial time.
  \end{definition}

  \begin{alertblock}{Reduction}
    Reduction is a way of converting one problem into another, so that a solution to one is a solution to the other.
    By reducing decision problem A to decision problem B, we mean that we can transform inputs to A into inputs to B in such a way that a given input to A is accepted iff the corresponding input to B is.
  \end{alertblock}

  \begin{definition}
    A problem is NP-complete if it's in NP and is NP-hard.
  \end{definition}
\end{frame}





