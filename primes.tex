\begin{frame}{PRIMES}
  \begin{definition}
    $$ PRIMES = \{ 2, 3, 5, 7, 11, 13, ... \} $$
  \end{definition}
  \begin{description}
    \item[PRIMES] is the set of all primes numbers.
    \item[Can] a Turing Machine be designed to decide PRIMES? (Yes)
    \item[Some] people say PRIMES is the decision problem for the set of primes.
    \item[Can] it do it in polynomial time? (Yes)
    \item[Is] PRIMES in P? (Yes, 2002)
  \end{description}
  \citeurl{www.cse.iitk.ac.in/users/manindra/algebra/primality\_v6.pdf}
\end{frame}

\begin{frame}{Modern cryptography}
  \begin{alertblock}{Modern asymmetric key cryptography is based on prime numbers}
    It depends on two facts:
    \begin{itemize}
      \item It's easy to verify primes (P).
      \item It's hard to decompose a composite number into primes (Not known to be P).
    \end{itemize}
  \end{alertblock}
  \begin{alertblock}{Generating versus verifying}
    Note that it's not necessarily easy to generate prime numbers.
    We know that verifying a number is prime can be done in polynomial time.
    That doesn't mean that we can generate prime numbers in polynomial time.
    You must start with the prime, and then ask the question.
  \end{alertblock}

\end{frame}

\begin{frame}[fragile]{Brute force prime checking}
  \begin{alertblock}{Is n a prime?}
    \begin{minted}{c}
function is_prime(n) {
  for (var i = 2; i < n; i++) {
    if (n % i == 0)
      return false;
  }
  return true;
}
    \end{minted}
  \end{alertblock}
\end{frame}

\begin{frame}[fragile]{More efficient prime checking}
  \begin{itemize}
    \item Suppose $a \times b = n$.
    \item Then $a < b$, $a > b$ or $a = b$.
    \item No matter what, $a \le \sqrt{n}$ and/or $b \le \sqrt{n}$.
    \item So only loop to $\sqrt{n}$.
    \item This still isn't that efficient.
  \end{itemize}
\end{frame}

\begin{frame}[fragile]{Slightly more efficient}
  \begin{alertblock}{Is n a prime?}
    \begin{minted}{c}
function is_prime(n) {
  for (var i = 2; i < Math.sqrt(n); i++) {
    if (n % i == 0)
      return false;
  }
  return true;
}
    \end{minted}
  \end{alertblock}
\end{frame}

%\begin{frame}[fragile]{AKS prime checking (2002)}
%  \begin{enumerate}
%    \item If $n = a^b$ where $a,b \in \mathbb{N}^+$, output composite.
%    \item Find the smallest $r$ such that $ord_r(n) > (log_2 n)^2$.
%    \item No matter what, $a \le \sqrt{n}$ and/or $b \le \sqrt{n}$.
%    \item So only loop to $\sqrt{n}$.
%    \item This still isn't that efficient.
%  \end{enumerate}
%  \citeurl{wikipedia.org/wiki/AKS\_primality\_test}
%\end{frame}